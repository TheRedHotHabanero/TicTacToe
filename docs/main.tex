
% "Крестики-нолики"

\documentclass[a5paper,10pt, twoside]{article} % тип документа

\usepackage{hyperref}
\usepackage{fancyhdr}
\usepackage{import}

% Математика

\import{headers/}{math.tex}

%  Русский язык

\import{headers/}{russian.tex}

% Дефайны

\import{headers/}{my_defs.tex}

%\fancyhf{}
\renewcommand{\footrulewidth}{ .0em }
\fancyfoot[C]{\texttt{\textemdash~\thepage~\textemdash}}
\fancyhead[L]{Tic-Tac-Toe}
\fancyhead[R]{}

\pagestyle{fancy}

\graphicspath{{src/pics/}} % где лежат картинки

\counterwithin{figure}{section}

% Title Page
\title
{
\hfill \break	\hfill \break
\hfill \break	\hfill \break
Попытка сделать что-то приличное

ОПИСАНИЕ ИГРЫ КРЕСТИКИ-НОЛИКИ
}
\author{Лирисман Карина, Б01-001}

%\setcounter{secnumdepth}{0}

\begin{document}

\maketitle


\thispagestyle{empty} % выключаем отображение номера для этой страницы

\newpage

\tableofcontents % Вывод содержания
\thispagestyle{plain}
\newpage


\section{Подготовительный этап работы}

  \subsection{Формулировка основных требований к игре}
  \import{src/}{1_req.tex}

  \subsection{Описание протокола игры}
  \import{src/}{2_protocol.tex}

  \subsection{Описание состояния игры}
  \import{src/}{3_cond.tex}
  
   \subsection{Уточнение протоколов деталями изменения состояния}
  \import{src/}{4_change.tex}



\section{Формализация}

	\subsection{Формализация структурной спецификации}

	\subsection{Формализация протоколов взаимодействия}


\section{Реализация}

\end{document}