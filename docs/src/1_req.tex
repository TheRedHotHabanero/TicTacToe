
\begin{enumerate}

	\item При запуске игры первым ходит первый игрок. После каждого тура очередность меняется и сбрасывается при перезапуске программы. 

	\item Какой игрок называется "первым", а какой "вторым", определяется по очередности подключения к серверу и фиксируется до конца работы программы или же до отключения одного из игроков от сервера.

	\item Победа присуждается в соответствии с классическими правилами игры "Крестики - нолики".
	
	\item Не предполагается вести счетчик побед каждого участника.
	
	\item Одновременно быть подключенными к серверу, а, соответственно, и вести игру, могут только два игрока.
	
	\item Нельзя начать игру или сделать ход, если хотя бы один из участников не подкючен к серверу.
	
	\item Каждый участник делает свой ход только после завершения хода другого участника.
	
	\item Всегда после хода "первого" игрока наступает ход "второго" игрока, если на ходе "первого" не случилось состояние победы или ничьи.
	
	\item Всегда после хода "второго" игрока наступает ход "первого" игрока, если на ходе "второго" не случилось состояние победы или ничьи.
	
	\item Каждый ход игрок должен поставить свою отметку на карте: "крестик" или "нолик". Ход считается выполненным только после того, как игрок поставил отметку.
	
	\item Никакой игрок не может делать свой ход во время выполнения хода другим игроком.
	
	\item "Первый" игрок всегда ходит "крестиками", а "второй" игрок всегда ходит "ноликами"
	
	\item Нельзя ставить свою отметку в клетку поля, в которой уже есть отметка этого же или другого игрока. Отметку можно ставить только в свободную клетку поля.
	
	\item Состояние завершения игры определяется ситуацией победы одного из игроков или же ситуацией, когда все клетки поля заняты, то есть ни один игрок не может сделать ход.
	
	\item В случае, если наступило состояние завершения игры без победы какого-либо из участников, присуждается ничья.
	
	\item После завершения каждого тура (игры) в консоли пишется сообщение о том, кто победил, или о состоянии ничьи.
	
	\item Первый тур начинается автоматически с задержкой в 10 секунд после подключения обоих игроков к серверу.
	
	\item При подключении к серверу пользователю показывается приветственное сообщение о том, что он успешно подключился.
	
	\item Когда на сервере 2 игрока, система оповещает обоих пользователей путем сообщений о том, что скоро начнется игра.
	
	\item Каждый тур, кроме первого, начинается после нажатия игроком на экране специальной кнопки или после введения специальной команды в консоли (пока не решено).
	
	\item Предусмотрено аварийное завершение игры, когда один из игроков или два игрока вместе специальным элементом управления игрой информируют систему о желании завершенить игру или пишут команду в консоли (пока не решено).
	
	\item В случае аварийного завершения игры на экране высвечивается сообщение для обоих игроков о том, что один из игроков хочет закончить игру", после чего с задержкой в 10 секунд программа завершает работу.
	
	\item Когда программа завершает работу, игроки отключаются от сервера, на экране пропадают все сообщения, закрываются все окна программы, очищаются ресурсы.

\end{enumerate}